% Generated Resume from YAML
% Auto-generated - Edit YAML files for content changes

\documentclass[
    a4paper,
    11pt,
]{resume}

\usepackage{ebgaramond}
\usepackage[UTF8]{ctex}
\usepackage{fontawesome}
\usepackage{xcolor}

\name{\heiti 才天一}
\address{ \kaishu {\color{gray}\faPhone\ \faWeixin}\ \ 18031521995 {} {} {} {} {\color{gray}\faEnvelope}\ \ tianyicai@vip.126.com }

\begin{document}

\begin{rSection}{教育背景}
    
    \textbf{耶鲁大学} \hfill \textit{2015 - 2016} \\ 
    {\kaishu 硕士,计算机科学,GPA: 4.0 / 4.0 } \hfill \textit{New Haven, CT}
    
    \textbf{亚利桑那大学} \hfill \textit{2011 - 2015} \\ 
    {\kaishu 本科,计算机科学,GPA: 4.0 / 4.0 } \hfill \textit{Tucson, AZ}

\end{rSection}

\begin{rSection}{工作经历}

    \begin{rSubsection}{Cobo}{2025/06 - 2025/08}{MCP-Now, 大模型应用开发}{北京}
        \item 负责开发 MCP-Now 后端基础架构,通过 Supabase Edge Functions 实现 PostgreSQL 数据库访问。构建第一版用户账户系统,包含安全认证、 注册流程和内容分享管理。使用 GitHub Actions 和 Supabase 分支集成功能 建立 CI/CD 流水线,实现自动化测试和部署。

        \item 利用 AI 开发工具(Claude Code、Cursor),提升开发效率和代码质量。
    \end{rSubsection}

    \begin{rSubsection}{摩尔线程}{2025/02 - 2025/05}{大模型训练平台, GPU 生态}{北京}
        \item 参与开发大语言模型训练平台,集成 Kubernetes、DL-Rover 和 Megatron-LM, 提供一键式部署、自动扩展和实时监控,支持大模型训练工作流。

        \item 参与开发高性能分布式内存系统,利用共享内存、RDMA 和 RoCE 网络技术, 为训练 checkpoint 在训练节点中提供缓存,减少节点故障恢复时间。
    \end{rSubsection}

    \begin{rSubsection}{字节跳动}{2021/10 - 2023/06}{抖音推荐, 离线架构}{北京}
        \item 开发并维护抖音近线数据流、视频索引和候选库。 开发基于 RocksDB 的 KV 存储索引写入模块,实现列式更新和行式读取的能力。

        \item 实现多租户场景下的全面资源监控和治理机制, 防止个别业务高负载影响主推荐数据流,避免索引存储空间过度消耗。

        \item 改进抖音和 TikTok 视频数据库架构,将单体索引拆解为视频和用户索引。 实现在线缓存命中率提升,增强用户信息一致性,同时简化社交等功能数据访问。

        \item 独立设计并实现了基于 Python 的轻量级流处理引擎, 成功部署于抖音团购等多个垂类业务场景。 核心能力:基于 \textit{epoll} 的事件驱动架构,实现高吞吐 IPC 通信; 利用 Kubernetes 原生的健康监控和自动故障恢复; 向后兼容历史业务代码同时方便业务同学开发迭代, 通过完全解耦垂类业务与主推荐信息流,实现特性迭代速度提升和可靠性。
    \end{rSubsection}

    \begin{rSubsection}{幻方量化}{2021/03 - 2021/08}{AI 训练平台(萤火超算集群), 存储与网络}{杭州}
        \item 开发基于 BeeGFS 的并行文件系统,通过 Linux Async I/O 优化异步写入性能, 提升读写吞吐。构建自动化测试环境,基于 GitLab CI/CD 实现虚拟机自动部署 和端到端测试,使用 InfluxDB 和 Grafana 实现性能测试结果可视化。

        \item 架构并实现基于 InfiniBand 网络的高性能 RDMA 异步通信库。 提供兼容 Boost.Asio 的 API 接口,基于 \textit{epoll} 实现事件驱动架构以获得 最佳性能。开发跨平台 Linux 和 Windows 原生版本,实现 Windows 应用与 Linux 存储集群的无缝集成。
    \end{rSubsection}

    \begin{rSubsection}{字节跳动}{2019/08 - 2020/07}{综合搜索, 在线架构}{北京}
        \item 维护并优化头条搜索在线服务,提供全网搜索和字节内部搜索平台功能。 涵盖召回、扇出控制、排序、请求控制、内容安全等核心模块。

        \item 治理下游服务依赖,统一服务注册、访问和监控机制。 通过多级缓存优化长尾服务响应,显著降低搜索延迟。

        \item 实施敏感内容标记流式更新机制,在检索早期阶段过滤风险内容, 保障内容安全的同时提升搜索质量。

        \item 重构 Xapian 索引库的元数据存储,将文档属性聚合为单一 Protobuf 结构, 减少检索阶段的元数据查询次数,使召回延迟降低约 10\%。

        \item 负责多个垂类搜索从 Elasticsearch 到自研 C++/Xapian/Hadoop 架构的迁移, 包括用户搜索、音乐搜索等。设计并实现支持大量异构小索引的平台架构方案。
    \end{rSubsection}

    \begin{rSubsection}{Google}{2018/03 - 2019/06}{Google Cloud 微服务治理, 全球软件负载均衡(GSLB)}{Sunnyvale, CA}
        \item 参与开发 GCP Traffic Director(托管式 Istio Pilot)产品。 该产品作为控制层向数据平面的 Envoy 代理提供服务网格配置, 实现服务发现、负载均衡和安全认证等核心功能。

        \item 设计并开发 Envoy 配置缓存、位置感知、Ingress 安全配置等核心组件。

        \item 设计 Envoy 代理在不同环境下的启动链接方式,涵盖 VM、GKE、 原生 Kubernetes,设计 Envoy GCP 和客户数据中心的部署方案。

        \item 开发基于 Python 的集成测试框架,通过读取配置自动调用 GCP API 构建测试环境,大幅减少手工操作,提升产品迭代效率。
    \end{rSubsection}

    \begin{rSubsection}{Datera, Inc}{2016/09 - 2018/02}{分布式块存储, 数据层}{Sunnyvale, CA}
        \item 参与开发基于 Linux B-Cache 和 iSCSI 协议的分布式块存储系统, 提供存储卷的多副本读写能力,支持自动横向扩展,支持 SSD HDD 混合机型。

        \item 维护数据层核心功能,包括副本间一致性检查与修复、 存储节点间负载均衡等关键特性。

        \item 参与设计数据加密、跨卷去重、本地数据中心到 AWS 远程备份等产品功能。
    \end{rSubsection}

\end{rSection}

\begin{rSection}{技术背景}

    \begin{tabular}{@{} >{\bfseries}l @{\hspace{6ex}} l @{}}
        编程语言 & C/C++, Python, Java \\
        系统架构 & 分布式存储, 流处理管道, 服务网格, 大模型训练 \\
        数据库 & RocksDB, PostgreSQL, 键值存储 \\
        基础设施 & Kubernetes, RDMA/InfiniBand, Linux 内核, CI/CD \\
    \end{tabular}

\end{rSection}

\vfill
\begin{center}
    \footnotesize\textit{由 Claude 协助编辑 \\
    源代码: https://github.com/tianyicaii/resume}
\end{center}

\end{document}
